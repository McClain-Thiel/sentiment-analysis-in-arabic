\documentclass[]{article}
\usepackage{lmodern}
\usepackage{amssymb,amsmath}
\usepackage{ifxetex,ifluatex}
\usepackage{arabtex}
\usepackage[utf8]{inputenc}
\usepackage{fixltx2e} % provides \textsubscript
\ifnum 0\ifxetex 1\fi\ifluatex 1\fi=0 % if pdftex
  \usepackage[T1]{fontenc}
  \usepackage[utf8]{inputenc}
\else % if luatex or xelatex
  \ifxetex
    \usepackage{mathspec}
  \else
    \usepackage{fontspec}
  \fi
  \defaultfontfeatures{Ligatures=TeX,Scale=MatchLowercase}
\fi
% use upquote if available, for straight quotes in verbatim environments
\IfFileExists{upquote.sty}{\usepackage{upquote}}{}
% use microtype if available
\IfFileExists{microtype.sty}{%
\usepackage{microtype}
\UseMicrotypeSet[protrusion]{basicmath} % disable protrusion for tt fonts
}{}
\usepackage[margin=1in]{geometry}
\usepackage{hyperref}
\hypersetup{unicode=true,
            pdftitle={Project\_final},
            pdfborder={0 0 0},
            breaklinks=true}
\urlstyle{same}  % don't use monospace font for urls
\usepackage{color}
\usepackage{fancyvrb}
\newcommand{\VerbBar}{|}
\newcommand{\VERB}{\Verb[commandchars=\\\{\}]}
\DefineVerbatimEnvironment{Highlighting}{Verbatim}{commandchars=\\\{\}}
% Add ',fontsize=\small' for more characters per line
\usepackage{framed}
\definecolor{shadecolor}{RGB}{248,248,248}
\newenvironment{Shaded}{\begin{snugshade}}{\end{snugshade}}
\newcommand{\AlertTok}[1]{\textcolor[rgb]{0.94,0.16,0.16}{#1}}
\newcommand{\AnnotationTok}[1]{\textcolor[rgb]{0.56,0.35,0.01}{\textbf{\textit{#1}}}}
\newcommand{\AttributeTok}[1]{\textcolor[rgb]{0.77,0.63,0.00}{#1}}
\newcommand{\BaseNTok}[1]{\textcolor[rgb]{0.00,0.00,0.81}{#1}}
\newcommand{\BuiltInTok}[1]{#1}
\newcommand{\CharTok}[1]{\textcolor[rgb]{0.31,0.60,0.02}{#1}}
\newcommand{\CommentTok}[1]{\textcolor[rgb]{0.56,0.35,0.01}{\textit{#1}}}
\newcommand{\CommentVarTok}[1]{\textcolor[rgb]{0.56,0.35,0.01}{\textbf{\textit{#1}}}}
\newcommand{\ConstantTok}[1]{\textcolor[rgb]{0.00,0.00,0.00}{#1}}
\newcommand{\ControlFlowTok}[1]{\textcolor[rgb]{0.13,0.29,0.53}{\textbf{#1}}}
\newcommand{\DataTypeTok}[1]{\textcolor[rgb]{0.13,0.29,0.53}{#1}}
\newcommand{\DecValTok}[1]{\textcolor[rgb]{0.00,0.00,0.81}{#1}}
\newcommand{\DocumentationTok}[1]{\textcolor[rgb]{0.56,0.35,0.01}{\textbf{\textit{#1}}}}
\newcommand{\ErrorTok}[1]{\textcolor[rgb]{0.64,0.00,0.00}{\textbf{#1}}}
\newcommand{\ExtensionTok}[1]{#1}
\newcommand{\FloatTok}[1]{\textcolor[rgb]{0.00,0.00,0.81}{#1}}
\newcommand{\FunctionTok}[1]{\textcolor[rgb]{0.00,0.00,0.00}{#1}}
\newcommand{\ImportTok}[1]{#1}
\newcommand{\InformationTok}[1]{\textcolor[rgb]{0.56,0.35,0.01}{\textbf{\textit{#1}}}}
\newcommand{\KeywordTok}[1]{\textcolor[rgb]{0.13,0.29,0.53}{\textbf{#1}}}
\newcommand{\NormalTok}[1]{#1}
\newcommand{\OperatorTok}[1]{\textcolor[rgb]{0.81,0.36,0.00}{\textbf{#1}}}
\newcommand{\OtherTok}[1]{\textcolor[rgb]{0.56,0.35,0.01}{#1}}
\newcommand{\PreprocessorTok}[1]{\textcolor[rgb]{0.56,0.35,0.01}{\textit{#1}}}
\newcommand{\RegionMarkerTok}[1]{#1}
\newcommand{\SpecialCharTok}[1]{\textcolor[rgb]{0.00,0.00,0.00}{#1}}
\newcommand{\SpecialStringTok}[1]{\textcolor[rgb]{0.31,0.60,0.02}{#1}}
\newcommand{\StringTok}[1]{\textcolor[rgb]{0.31,0.60,0.02}{#1}}
\newcommand{\VariableTok}[1]{\textcolor[rgb]{0.00,0.00,0.00}{#1}}
\newcommand{\VerbatimStringTok}[1]{\textcolor[rgb]{0.31,0.60,0.02}{#1}}
\newcommand{\WarningTok}[1]{\textcolor[rgb]{0.56,0.35,0.01}{\textbf{\textit{#1}}}}
\usepackage{longtable,booktabs}
\usepackage{graphicx,grffile}
\makeatletter
\def\maxwidth{\ifdim\Gin@nat@width>\linewidth\linewidth\else\Gin@nat@width\fi}
\def\maxheight{\ifdim\Gin@nat@height>\textheight\textheight\else\Gin@nat@height\fi}
\makeatother
% Scale images if necessary, so that they will not overflow the page
% margins by default, and it is still possible to overwrite the defaults
% using explicit options in \includegraphics[width, height, ...]{}
\setkeys{Gin}{width=\maxwidth,height=\maxheight,keepaspectratio}
\IfFileExists{parskip.sty}{%
\usepackage{parskip}
}{% else
\setlength{\parindent}{0pt}
\setlength{\parskip}{6pt plus 2pt minus 1pt}
}
\setlength{\emergencystretch}{3em}  % prevent overfull lines
\providecommand{\tightlist}{%
  \setlength{\itemsep}{0pt}\setlength{\parskip}{0pt}}
\setcounter{secnumdepth}{0}
% Redefines (sub)paragraphs to behave more like sections
\ifx\paragraph\undefined\else
\let\oldparagraph\paragraph
\renewcommand{\paragraph}[1]{\oldparagraph{#1}\mbox{}}
\fi
\ifx\subparagraph\undefined\else
\let\oldsubparagraph\subparagraph
\renewcommand{\subparagraph}[1]{\oldsubparagraph{#1}\mbox{}}
\fi

%%% Use protect on footnotes to avoid problems with footnotes in titles
\let\rmarkdownfootnote\footnote%
\def\footnote{\protect\rmarkdownfootnote}

%%% Change title format to be more compact
\usepackage{titling}

% Create subtitle command for use in maketitle
\providecommand{\subtitle}[1]{
  \posttitle{
    \begin{center}\large#1\end{center}
    }
}

\setlength{\droptitle}{-2em}

  \title{Project\_final}
    \pretitle{\vspace{\droptitle}\centering\huge}
  \posttitle{\par}
    \author{}
    \preauthor{}\postauthor{}
    \date{}
    \predate{}\postdate{}
  

\begin{document}
\maketitle

First read in all text and clean and translate/map to a common
repersentation. I don't know the mapping of the arabic words to the
transliterated ones but the library is able to shift back and forth.This
block reads all text in, removes the stop words, and stems and
transliterates the text to latin characters and then pputs it back into
the data frame

I first ran this analysis in python which has diffrent cleaning and
stemming libraries, so if the report differs from the results in some
ways that may be the root. For functional purpouses it should be the
same.

\begin{Shaded}
\begin{Highlighting}[]
\KeywordTok{library}\NormalTok{(knitr)}
\NormalTok{master =}\StringTok{ }\KeywordTok{read.csv}\NormalTok{(}\StringTok{'data.csv'}\NormalTok{, }\DataTypeTok{stringsAsFactors =} \OtherTok{FALSE}\NormalTok{)}
\NormalTok{master =}\StringTok{ }\KeywordTok{as.data.frame}\NormalTok{(master)}
\KeywordTok{typeof}\NormalTok{(master)}
\end{Highlighting}
\end{Shaded}

{[}1{]} ``list''

\begin{Shaded}
\begin{Highlighting}[]
\KeywordTok{library}\NormalTok{(arabicStemR)}
\end{Highlighting}
\end{Shaded}

\begin{verbatim}
## 
## Attaching package: 'arabicStemR'
\end{verbatim}

\begin{verbatim}
## The following object is masked from 'package:graphics':
## 
##     stem
\end{verbatim}

\begin{Shaded}
\begin{Highlighting}[]
\NormalTok{wrap =}\StringTok{ }\ControlFlowTok{function}\NormalTok{(chr)\{}
  \KeywordTok{return}\NormalTok{(}\KeywordTok{as.character}\NormalTok{(}\KeywordTok{removeStopWords}\NormalTok{(chr)[}\DecValTok{1}\NormalTok{]))}
\NormalTok{\}}
\NormalTok{master}\OperatorTok{$}\NormalTok{transliterated =}\StringTok{ }\KeywordTok{removePunctuation}\NormalTok{(}\KeywordTok{cleanChars}\NormalTok{(}\KeywordTok{transliterate}\NormalTok{(}\KeywordTok{sapply}\NormalTok{(master}\OperatorTok{$}\NormalTok{Text, wrap))))}
\KeywordTok{kable}\NormalTok{(master[}\DecValTok{1}\OperatorTok{:}\DecValTok{5}\NormalTok{,])}
\end{Highlighting}
\end{Shaded}

\begin{longtable}[]{@{}rlrl@{}}
\toprule
X & Text & Positive. & transliterated\tabularnewline
\midrule
\endhead
0 & أبذل الجهد و العرق و توكل على الله و اطمئن فرزقك مضمون & 1 & abil
aljhd al3rQ twkl allh aTm5n frzQk mDmwn\tabularnewline
1 & احُبك ربي & 1 & a7bk rby\tabularnewline
2 & اللهم عفوك و رضاك و الجنه & 1 & allhm 3fwk rDak aljnh\tabularnewline
3 & الماء هى الحياة & 1 & almaq hA al7ya0\tabularnewline
4 & بالفعل لازم كل واحد بيتكلم عن الدين اقل مافيها يلتزم باقل مبادئه
عشان يكون قدوة مش ييخلي الناس تقول هو ده الدين & 1 & balf3l lazm kl wa7d
bytklm aldyn mafyha yltzm baQl mbad5h 3Wan ykwn Qdw0 mW yyKly alnas tQwl
dh aldyn\tabularnewline
I fir & st ran this analysis in python which has diffrent cleaning and
stemming libraries, so if the report differs from t & he results i & n
some ways that may be the root. For functional purpouses it should be
the same.\tabularnewline
\bottomrule
\end{longtable}

Spliting data in to train and test sets, building word frequency tables
for the positive and negitive words and then building training and
testing data to feed into naive bayes.

\begin{Shaded}
\begin{Highlighting}[]
\KeywordTok{library}\NormalTok{(keras)}
\KeywordTok{library}\NormalTok{(stringr)}

\NormalTok{word_counts <-}\StringTok{ }\KeywordTok{as.data.frame}\NormalTok{(}\KeywordTok{table}\NormalTok{(}\KeywordTok{unlist}\NormalTok{( }\KeywordTok{strsplit}\NormalTok{(master}\OperatorTok{$}\NormalTok{transliterated, }\StringTok{"\textbackslash{} "}\NormalTok{) )))  }
\NormalTok{word_counts}\OperatorTok{$}\NormalTok{id =}\StringTok{ }\KeywordTok{seq}\NormalTok{(}\DecValTok{1}\NormalTok{,}\KeywordTok{nrow}\NormalTok{(word_counts))}

\NormalTok{get_int_rep =}\StringTok{ }\ControlFlowTok{function}\NormalTok{(string)\{}
\NormalTok{  int_rep =}\StringTok{ }\KeywordTok{c}\NormalTok{()}
  \ControlFlowTok{for}\NormalTok{ (x }\ControlFlowTok{in} \KeywordTok{unlist}\NormalTok{(}\KeywordTok{strsplit}\NormalTok{(string, }\StringTok{' '}\NormalTok{)))\{}
    \ControlFlowTok{if}\NormalTok{ ((x }\OperatorTok\StringTok{ }\KeywordTok{unlist}\NormalTok{(word_counts[}\StringTok{'Var1'}\NormalTok{])))\{}
\NormalTok{      row =}\StringTok{ }\NormalTok{word_counts[word_counts[}\StringTok{'Var1'}\NormalTok{] }\OperatorTok{==}\StringTok{ }\NormalTok{x,]}
\NormalTok{      int_rep =}\StringTok{ }\KeywordTok{append}\NormalTok{(int_rep, row}\OperatorTok{$}\NormalTok{id)}
\NormalTok{    \}}
\NormalTok{  \}}
  \ControlFlowTok{if}\NormalTok{(}\KeywordTok{length}\NormalTok{(int_rep) }\OperatorTok{>}\StringTok{ }\DecValTok{1}\NormalTok{)\{}\KeywordTok{return}\NormalTok{(}\KeywordTok{array}\NormalTok{(}\KeywordTok{pad_sequences}\NormalTok{(}\KeywordTok{list}\NormalTok{(int_rep), }\DataTypeTok{maxlen =} \DecValTok{180}\NormalTok{)))\}}
  \ControlFlowTok{else}\NormalTok{\{}\KeywordTok{return}\NormalTok{(}\KeywordTok{array}\NormalTok{(}\KeywordTok{pad_sequences}\NormalTok{(}\KeywordTok{list}\NormalTok{(}\DecValTok{0}\NormalTok{), }\DataTypeTok{maxlen =} \DecValTok{180}\NormalTok{)))\}}
\NormalTok{\}}

\NormalTok{int_reps =}\StringTok{ }\KeywordTok{c}\NormalTok{()}
\ControlFlowTok{for}\NormalTok{ (x }\ControlFlowTok{in}\NormalTok{ master}\OperatorTok{$}\NormalTok{transliterated)\{}
\NormalTok{  int_reps =}\StringTok{ }\KeywordTok{append}\NormalTok{(int_reps, }\KeywordTok{get_int_rep}\NormalTok{(x))}
\NormalTok{\}}

\NormalTok{t =}\StringTok{ }\KeywordTok{array_reshape}\NormalTok{(int_reps, }\KeywordTok{c}\NormalTok{(}\KeywordTok{nrow}\NormalTok{(master), }\DecValTok{180}\NormalTok{))}
\NormalTok{master}\OperatorTok{$}\NormalTok{int_rep =}\StringTok{ }\NormalTok{t}

\NormalTok{smp_size <-}\StringTok{ }\KeywordTok{floor}\NormalTok{(}\FloatTok{0.75} \OperatorTok{*}\StringTok{ }\KeywordTok{nrow}\NormalTok{(master))}
\NormalTok{train_ind <-}\StringTok{ }\KeywordTok{sample}\NormalTok{(}\KeywordTok{seq_len}\NormalTok{(}\KeywordTok{nrow}\NormalTok{(master)), }\DataTypeTok{size =}\NormalTok{ smp_size)}
\NormalTok{train <-}\StringTok{ }\NormalTok{master[train_ind, ]}
\NormalTok{test <-}\StringTok{ }\NormalTok{master[}\OperatorTok{-}\NormalTok{train_ind, ]}

\NormalTok{pos_words =}\StringTok{ }\NormalTok{train[train}\OperatorTok{$}\NormalTok{Positive. }\OperatorTok{==}\StringTok{ }\DecValTok{1}\NormalTok{,]}
\NormalTok{neg_words =}\StringTok{ }\NormalTok{train[train}\OperatorTok{$}\NormalTok{Positive. }\OperatorTok{==}\StringTok{ }\DecValTok{0}\NormalTok{,]}

\NormalTok{pos_word_tbl =}\StringTok{ }\KeywordTok{as.data.frame}\NormalTok{(}\KeywordTok{table}\NormalTok{(}\KeywordTok{unlist}\NormalTok{( }\KeywordTok{strsplit}\NormalTok{(pos_words}\OperatorTok{$}\NormalTok{transliterated, }\StringTok{"\textbackslash{} "}\NormalTok{) )))}
\NormalTok{neg_word_tbl =}\StringTok{ }\KeywordTok{as.data.frame}\NormalTok{(}\KeywordTok{table}\NormalTok{(}\KeywordTok{unlist}\NormalTok{( }\KeywordTok{strsplit}\NormalTok{(neg_words}\OperatorTok{$}\NormalTok{transliterated, }\StringTok{"\textbackslash{} "}\NormalTok{) )))}

\NormalTok{total_words =}\StringTok{ }\KeywordTok{sum}\NormalTok{(word_counts}\OperatorTok{$}\NormalTok{Freq)}

\NormalTok{word_counts}\OperatorTok{$}\NormalTok{log_frequency =}\StringTok{ }\KeywordTok{log}\NormalTok{(word_counts}\OperatorTok{$}\NormalTok{Freq}\OperatorTok{/}\NormalTok{total_words)}
\NormalTok{pos_word_tbl}\OperatorTok{$}\NormalTok{log_frequency =}\StringTok{ }\KeywordTok{log}\NormalTok{(pos_word_tbl}\OperatorTok{$}\NormalTok{Freq}\OperatorTok{/}\NormalTok{total_words)}
\NormalTok{neg_word_tbl}\OperatorTok{$}\NormalTok{log_frequency =}\StringTok{ }\KeywordTok{log}\NormalTok{(neg_word_tbl}\OperatorTok{$}\NormalTok{Freq}\OperatorTok{/}\NormalTok{total_words)}

\KeywordTok{kable}\NormalTok{(word_counts[}\DecValTok{1}\OperatorTok{:}\DecValTok{5}\NormalTok{,], }\DataTypeTok{caption =} \StringTok{'Word frequency Tables'}\NormalTok{)}
\end{Highlighting}
\end{Shaded}

\begin{longtable}[]{@{}lrrr@{}}
\caption{Word frequency Tables}\tabularnewline
\toprule
Var1 & Freq & id & log\_frequency\tabularnewline
\midrule
\endfirsthead
\toprule
Var1 & Freq & id & log\_frequency\tabularnewline
\midrule
\endhead
0 & 1 & 1 & -9.537123\tabularnewline
1 & 3 & 2 & -8.438511\tabularnewline
10 & 3 & 3 & -8.438511\tabularnewline
١٠٠ & 2 & 4 & -8.843976\tabularnewline
100 & 4 & 5 & -8.150829\tabularnewline
\bottomrule
\end{longtable}

\begin{Shaded}
\begin{Highlighting}[]
\KeywordTok{kable}\NormalTok{(pos_word_tbl[}\DecValTok{1}\OperatorTok{:}\DecValTok{5}\NormalTok{,], }\DataTypeTok{caption =}  \StringTok{'Positive words and thier frequency'}\NormalTok{)}
\end{Highlighting}
\end{Shaded}

\begin{longtable}[]{@{}lrr@{}}
\caption{Positive words and thier frequency}\tabularnewline
\toprule
Var1 & Freq & log\_frequency\tabularnewline
\midrule
\endfirsthead
\toprule
Var1 & Freq & log\_frequency\tabularnewline
\midrule
\endhead
3 & 4 & -8.150829\tabularnewline
3afyt & 1 & -9.537123\tabularnewline
3afytk & 1 & -9.537123\tabularnewline
3almyn & 1 & -9.537123\tabularnewline
3aly & 1 & -9.537123\tabularnewline
\bottomrule
\end{longtable}

Defining slightly modifyed naive bayes function

\begin{Shaded}
\begin{Highlighting}[]
\NormalTok{naive_b =}\StringTok{ }\ControlFlowTok{function}\NormalTok{(string)\{}
\NormalTok{  prob_p =}\StringTok{ }\DecValTok{1}
\NormalTok{  prob_n =}\StringTok{ }\DecValTok{1}
\NormalTok{  multiplier =}\StringTok{ }\FloatTok{.5}
  \ControlFlowTok{for}\NormalTok{ (x }\ControlFlowTok{in} \KeywordTok{unlist}\NormalTok{(}\KeywordTok{strsplit}\NormalTok{(string, }\StringTok{' '}\NormalTok{)))\{}
    \ControlFlowTok{if}\NormalTok{ (x }\OperatorTok\StringTok{ }\KeywordTok{unlist}\NormalTok{(pos_word_tbl[}\StringTok{'Var1'}\NormalTok{]))\{}
\NormalTok{      freq =}\StringTok{ }\NormalTok{pos_word_tbl[pos_word_tbl[}\StringTok{'Var1'}\NormalTok{] }\OperatorTok{==}\StringTok{ }\NormalTok{x,]}
\NormalTok{      prob_p =}\StringTok{ }\NormalTok{prob_p }\OperatorTok{+}\StringTok{ }\NormalTok{freq}\OperatorTok{$}\NormalTok{log_frequency}
\NormalTok{    \}}
    \ControlFlowTok{if}\NormalTok{ (x }\OperatorTok\StringTok{ }\KeywordTok{unlist}\NormalTok{(neg_word_tbl[}\StringTok{'Var1'}\NormalTok{]))\{}
\NormalTok{      freq =}\StringTok{ }\NormalTok{neg_word_tbl[neg_word_tbl[}\StringTok{'Var1'}\NormalTok{] }\OperatorTok{==}\StringTok{ }\NormalTok{x,]}
\NormalTok{      prob_n =}\StringTok{ }\NormalTok{prob_n }\OperatorTok{+}\StringTok{ }\NormalTok{freq}\OperatorTok{$}\NormalTok{log_frequency}
\NormalTok{    \}}
\NormalTok{  \}}
  \ControlFlowTok{if}\NormalTok{ (prob_n }\OperatorTok{>}\StringTok{ }\NormalTok{prob_p)\{}\KeywordTok{return}\NormalTok{(}\DecValTok{1}\NormalTok{)\}}
  \ControlFlowTok{else}\NormalTok{\{}\KeywordTok{return}\NormalTok{(}\DecValTok{0}\NormalTok{)\}}
\NormalTok{\}}
\end{Highlighting}
\end{Shaded}

Running the naive bayes on training and test data

\begin{Shaded}
\begin{Highlighting}[]
\NormalTok{train}\OperatorTok{$}\NormalTok{guesses =}\StringTok{ }\KeywordTok{sapply}\NormalTok{(}\KeywordTok{as.list}\NormalTok{(train}\OperatorTok{$}\NormalTok{transliterated), naive_b)}
\NormalTok{test}\OperatorTok{$}\NormalTok{guesses =}\StringTok{ }\KeywordTok{sapply}\NormalTok{(}\KeywordTok{as.list}\NormalTok{(test}\OperatorTok{$}\NormalTok{transliterated), naive_b)}
\end{Highlighting}
\end{Shaded}

Evaluation of results Note: The accuracy of this test is suspiciously
good. I'm not sure if this is a flux or if the transliteration function
is much more complex than my solution, but I used the accruacies from my
original code in python in the report because I know the mapping
function and am much more confident in the reproducability of my
solution. If I had more time I would be able ot verify the library's
transliteration function and these results but until this I'll only
report on what I know to be accurate.

\begin{Shaded}
\begin{Highlighting}[]
\NormalTok{train_accruacy =}\StringTok{ }\KeywordTok{sum}\NormalTok{(}\KeywordTok{ifelse}\NormalTok{(train}\OperatorTok{$}\NormalTok{Positive.}\OperatorTok{==}\NormalTok{train}\OperatorTok{$}\NormalTok{guesses,}\DecValTok{1}\NormalTok{,}\DecValTok{0}\NormalTok{)) }\OperatorTok{/}\StringTok{ }\KeywordTok{nrow}\NormalTok{(train)}
\NormalTok{test_accuracy =}\StringTok{ }\KeywordTok{sum}\NormalTok{(}\KeywordTok{ifelse}\NormalTok{(test}\OperatorTok{$}\NormalTok{Positive.}\OperatorTok{==}\NormalTok{test}\OperatorTok{$}\NormalTok{guesses,}\DecValTok{1}\NormalTok{,}\DecValTok{0}\NormalTok{)) }\OperatorTok{/}\StringTok{ }\KeywordTok{nrow}\NormalTok{(test)}
\KeywordTok{print}\NormalTok{(}\StringTok{'Train accuracy for modified naive bayes is :'}\NormalTok{)}
\end{Highlighting}
\end{Shaded}

\begin{verbatim}
## [1] "Train accuracy for modified naive bayes is :"
\end{verbatim}

\begin{Shaded}
\begin{Highlighting}[]
\NormalTok{train_accruacy}
\end{Highlighting}
\end{Shaded}

\begin{verbatim}
## [1] 0.9827586
\end{verbatim}

\begin{Shaded}
\begin{Highlighting}[]
\KeywordTok{print}\NormalTok{(}\StringTok{'Test accuracy for modified naive bayes is :'}\NormalTok{)}
\end{Highlighting}
\end{Shaded}

\begin{verbatim}
## [1] "Test accuracy for modified naive bayes is :"
\end{verbatim}

\begin{Shaded}
\begin{Highlighting}[]
\NormalTok{test_accuracy}
\end{Highlighting}
\end{Shaded}

\begin{verbatim}
## [1] 0.8031809
\end{verbatim}

This chunk implements the RNN. I added a dropout layer to combat
overfitting but it doesn't help as much as I would like. I belive the
encoding is less consistent with the R version of this code because when
I build a RNN with the same architechure, the python version
outperformed by almost 20\% and was far less volitile.

\begin{Shaded}
\begin{Highlighting}[]
\KeywordTok{library}\NormalTok{(keras)}
\KeywordTok{library}\NormalTok{(tensorflow)}
\CommentTok{#install_tensorflow(version =  '1.12') #recent version is broken...?}
\NormalTok{embedding_size =}\StringTok{ }\DecValTok{32}
\NormalTok{vocab_size =}\StringTok{ }\KeywordTok{nrow}\NormalTok{(word_counts)}
\NormalTok{max_words =}\StringTok{ }\DecValTok{180}

\NormalTok{model =}\StringTok{ }\KeywordTok{keras_model_sequential}\NormalTok{()}
\NormalTok{model }\OperatorTok
\StringTok{  }\KeywordTok{layer_embedding}\NormalTok{(vocab_size, }\DecValTok{64}\NormalTok{, }\DataTypeTok{input_length =}\NormalTok{ max_words, }\DataTypeTok{name =} \StringTok{'embedding'}\NormalTok{) }\OperatorTok
\StringTok{  }\KeywordTok{bidirectional}\NormalTok{(}\KeywordTok{layer_lstm}\NormalTok{(}\DataTypeTok{units =}\DecValTok{64}\NormalTok{, }\DataTypeTok{name =} \StringTok{'reccurent_layer'}\NormalTok{)) }\OperatorTok
\StringTok{  }\KeywordTok{layer_dropout}\NormalTok{(.}\DecValTok{5}\NormalTok{) }\OperatorTok
\StringTok{  }\KeywordTok{layer_dense}\NormalTok{(}\DecValTok{1}\NormalTok{, }\DataTypeTok{activation =} \StringTok{'sigmoid'}\NormalTok{, }\DataTypeTok{name =} \StringTok{'output'}\NormalTok{)}

\NormalTok{model }\OperatorTok\StringTok{ }\KeywordTok{compile}\NormalTok{(}
  \DataTypeTok{optimizer =} \StringTok{'adam'}\NormalTok{,}
  \DataTypeTok{loss =} \StringTok{'binary_crossentropy'}\NormalTok{,}
  \DataTypeTok{metrics =} \KeywordTok{c}\NormalTok{(}\StringTok{'accuracy'}\NormalTok{)}
\NormalTok{)}
\KeywordTok{summary}\NormalTok{(model)}
\end{Highlighting}
\end{Shaded}

\begin{verbatim}
## ___________________________________________________________________________
## Layer (type)                     Output Shape                  Param #     
## ===========================================================================
## embedding (Embedding)            (None, 180, 64)               426368      
## ___________________________________________________________________________
## bidirectional (Bidirectional)    (None, 128)                   66048       
## ___________________________________________________________________________
## dropout (Dropout)                (None, 128)                   0           
## ___________________________________________________________________________
## output (Dense)                   (None, 1)                     129         
## ===========================================================================
## Total params: 492,545
## Trainable params: 492,545
## Non-trainable params: 0
## ___________________________________________________________________________
\end{verbatim}

\begin{Shaded}
\begin{Highlighting}[]
\NormalTok{X_valid =}\StringTok{ }\NormalTok{train}\OperatorTok{$}\NormalTok{int_rep}
\NormalTok{y_val =}\StringTok{ }\NormalTok{train}\OperatorTok{$}\NormalTok{Positive.}
\NormalTok{X_train2 =}\StringTok{ }\NormalTok{test}\OperatorTok{$}\NormalTok{int_rep}
\NormalTok{y_train2 =}\StringTok{ }\NormalTok{test}\OperatorTok{$}\NormalTok{Positive.}

\NormalTok{history <-}\StringTok{ }\NormalTok{model }\OperatorTok\StringTok{ }\KeywordTok{fit}\NormalTok{(}
\NormalTok{  X_train2, y_train2, }
  \DataTypeTok{epochs =} \DecValTok{35}\NormalTok{, }\DataTypeTok{batch_size =}\DecValTok{64}\NormalTok{, }
  \DataTypeTok{validation_split =} \FloatTok{0.2}
\NormalTok{)}
\end{Highlighting}
\end{Shaded}

This chunk uses a hashmap to find a euclidian distance be occurance of
individual words. Hashmap is used because its crazy fat compared to
whatever R lists and columns are made out of. For more details and
exploratory data analysis check the report and the python document.

\begin{Shaded}
\begin{Highlighting}[]
\CommentTok{#install.packages('hashmap')}
\KeywordTok{library}\NormalTok{(hashmap)}
\NormalTok{word_c =}\StringTok{ }\KeywordTok{hashmap}\NormalTok{(}\KeywordTok{as.character}\NormalTok{(word_counts}\OperatorTok{$}\NormalTok{Var1), word_counts}\OperatorTok{$}\NormalTok{id )}
\NormalTok{find_location =}\StringTok{ }\ControlFlowTok{function}\NormalTok{(string)\{}
\NormalTok{  map =}\StringTok{ }\KeywordTok{hashmap}\NormalTok{(}\KeywordTok{seq}\NormalTok{(}\DecValTok{1}\NormalTok{, }\KeywordTok{nrow}\NormalTok{(word_counts)), }\KeywordTok{numeric}\NormalTok{(}\KeywordTok{nrow}\NormalTok{(word_counts)))}
  \ControlFlowTok{for}\NormalTok{ (x }\ControlFlowTok{in} \KeywordTok{unlist}\NormalTok{(}\KeywordTok{strsplit}\NormalTok{(string, }\StringTok{' '}\NormalTok{)))\{}
\NormalTok{      x =}\StringTok{ }\KeywordTok{as.character}\NormalTok{(x)}
      \ControlFlowTok{if}\NormalTok{ (word_c}\OperatorTok{$}\KeywordTok{has_key}\NormalTok{(x) }\OperatorTok{==}\StringTok{ }\OtherTok{TRUE}\NormalTok{)\{}
\NormalTok{        id =}\StringTok{ }\NormalTok{word_c[[x]]}
\NormalTok{        map[[id]] =}\StringTok{ }\NormalTok{map[[id]] }\OperatorTok{+}\StringTok{ }\DecValTok{1}
\NormalTok{    \}}
\NormalTok{  \}}
  \KeywordTok{return}\NormalTok{(}\KeywordTok{c}\NormalTok{(map}\OperatorTok{$}\KeywordTok{values}\NormalTok{()))}
\NormalTok{\}}

\NormalTok{data =}\StringTok{ }\KeywordTok{c}\NormalTok{()}
\NormalTok{num =}\StringTok{ }\DecValTok{1}
\ControlFlowTok{for}\NormalTok{ (x }\ControlFlowTok{in}\NormalTok{ master}\OperatorTok{$}\NormalTok{transliterated)\{}
\NormalTok{  data =}\StringTok{ }\KeywordTok{append}\NormalTok{(data, }\KeywordTok{find_location}\NormalTok{(x))}
\NormalTok{\} }

\NormalTok{data2 =}\StringTok{ }\KeywordTok{array_reshape}\NormalTok{(data, }\DataTypeTok{dim =}\KeywordTok{c}\NormalTok{(}\DecValTok{2011}\NormalTok{,}\KeywordTok{length}\NormalTok{(}\KeywordTok{find_location}\NormalTok{(master}\OperatorTok{$}\NormalTok{transliterated[}\DecValTok{1}\NormalTok{]))))}

\KeywordTok{kable}\NormalTok{(master[}\DecValTok{1}\OperatorTok{:}\DecValTok{5}\NormalTok{,], }\DataTypeTok{caption =} \StringTok{'master with grouping variable'}\NormalTok{)}
\end{Highlighting}
\end{Shaded}

\begin{verbatim}
## Warning in `[<-.data.frame`(`*tmp*`, , isn, value = structure(list(X =
## structure(c("0", : provided 182 variables to replace 3 variables
\end{verbatim}

\begin{longtable}[]{@{}rlrlr@{}}
\caption{master with grouping variable}\tabularnewline
\toprule
X & Text & Positive. & transliterated & int\_rep\tabularnewline
\midrule
\endfirsthead
\toprule
X & Text & Positive. & transliterated & int\_rep\tabularnewline
\midrule
\endhead
0 & أبذل الجهد و العرق و توكل على الله و اطمئن فرزقك مضمون & 1 & abil
aljhd al3rQ twkl allh aTm5n frzQk mDmwn & 0\tabularnewline
1 & احُبك ربي & 1 & a7bk rby & 0\tabularnewline
2 & اللهم عفوك و رضاك و الجنه & 1 & allhm 3fwk rDak aljnh &
0\tabularnewline
3 & الماء هى الحياة & 1 & almaq hA al7ya0 & 0\tabularnewline
4 & بالفعل لازم كل واحد بيتكلم عن الدين اقل مافيها يلتزم باقل مبادئه
عشان يكون قدوة مش ييخلي الناس تقول هو ده الدين & 1 & balf3l lazm kl wa7d
bytklm aldyn mafyha yltzm baQl mbad5h 3Wan ykwn Qdw0 mW yyKly alnas tQwl
dh aldyn & 0\tabularnewline
\bottomrule
\end{longtable}

This chunk calculated the k-means groupings and appends it to the master
dataframe which associates it with a tweet. I ran this many times and
looked through the data manually but the code takes up a lot of spcae so
this is just one itteration.

\begin{Shaded}
\begin{Highlighting}[]
\NormalTok{kmean =}\StringTok{ }\KeywordTok{kmeans}\NormalTok{(data2, }\DecValTok{5}\NormalTok{)}
\NormalTok{master}\OperatorTok{$}\NormalTok{group_}\DecValTok{5}\NormalTok{ =}\StringTok{ }\NormalTok{kmean}\OperatorTok{$}\NormalTok{cluster}
\KeywordTok{kable}\NormalTok{(master[}\DecValTok{1}\OperatorTok{:}\DecValTok{5}\NormalTok{,], }\DataTypeTok{caption =} \StringTok{'master with grouping variable'}\NormalTok{)}
\end{Highlighting}
\end{Shaded}

\begin{verbatim}
## Warning in `[<-.data.frame`(`*tmp*`, , isn, value = structure(list(X =
## structure(c("0", : provided 183 variables to replace 4 variables
\end{verbatim}

\begin{longtable}[]{@{}rlrlrr@{}}
\caption{master with grouping variable}\tabularnewline
\toprule
X & Text & Positive. & transliterated & int\_rep &
group\_5\tabularnewline
\midrule
\endfirsthead
\toprule
X & Text & Positive. & transliterated & int\_rep &
group\_5\tabularnewline
\midrule
\endhead
0 & أبذل الجهد و العرق و توكل على الله و اطمئن فرزقك مضمون & 1 & abil
aljhd al3rQ twkl allh aTm5n frzQk mDmwn & 0 & 0\tabularnewline
1 & احُبك ربي & 1 & a7bk rby & 0 & 0\tabularnewline
2 & اللهم عفوك و رضاك و الجنه & 1 & allhm 3fwk rDak aljnh & 0 &
0\tabularnewline
3 & الماء هى الحياة & 1 & almaq hA al7ya0 & 0 & 0\tabularnewline
4 & بالفعل لازم كل واحد بيتكلم عن الدين اقل مافيها يلتزم باقل مبادئه
عشان يكون قدوة مش ييخلي الناس تقول هو ده الدين & 1 & balf3l lazm kl wa7d
bytklm aldyn mafyha yltzm baQl mbad5h 3Wan ykwn Qdw0 mW yyKly alnas tQwl
dh aldyn & 0 & 0\tabularnewline
\bottomrule
\end{longtable}


\end{document}
